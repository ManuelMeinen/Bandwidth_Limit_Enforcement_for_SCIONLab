\chapter{Conclusion}
\section{Front End}

Implementing the front end was the easiest part of the entire project. For that, I stuck mostly to the current practices and implemented the mechanisms that were additionally used for this project in a manner that is very similar to the mechanisms that were already in place. The intention of this approach was to make maintainability as easy as possible, especially since the \acs{SCIONLab} server is still in development. I also made sure that the code that I contributed was well documented and easy to read.

\section{Back End}

Designing and implementing the \textit{scionlab\_bw\_limiter} was definitely the part of the project that required the most research. Understanding and configuring \acs{TC} turned out to be more challenging than expected. Furthermore it is rather unfortunate that there isn't really a reasonable \acs{API} that lets us configure \acs{TC} using Python. Therefore I had to write my own wrapper for \acs{TC}. But I think that the implementation of the \textit{scionlab\_bw\_limiter} works fairly well and is also easy to understand. Even though the results the \textit{scionlab\_bw\_limiter} is delivering are not impeccable, they are still satisfying. Especially by considering the fact that, as far as I know, no better performing tool exists. In particular the results for ingress traffic were at first rather troubling. 
But after revealing the cause behind those strange results, I consider my results to be rather satisfying. 
The fact that the well established bandwidth limiter \textit{wondershaper} shows similar results, confirms my findings.

\newpage

\section{Difficulties that Arose}

The biggest challenge I was facing, was enforcing a bandwidth limit on ingress traffic. \acs{TC} was mainly designed to do traffic shaping in order to improve the quality of service. For improving the quality of service, egress traffic is much more relevant than ingress traffic. Therefore \acs{TC} is not as advanced, when it comes to handling ingress traffic, as I wished it would be.
\\
Furthermore, testing turned out to be quite difficult as well. It's not easy to create a realistic test environment, which is still simple enough such that the test results can be interpreted in a meaningful way. If test results turn out differently than expected, it is often not clear whether the measurement was wrong, the test configuration was faulty, the test environment was unrealistic or the implementation was buggy. It might as well just be that the testing tool is limited in its measurement capabilities. 

\section{Lessons Learned}

This bachelor thesis project offered me the opportunity to gain a deep understanding of how \acs{IP} traffic can be managed in order to enforce a bandwidth limit and what possible complications and consequences might occur. It also provided me with a holistic view over \acs{SCION}, \acs{SCIONLab} and in general how a novel internet architecture can be tested in a distributed testbed. Furthermore, I acquired a better understanding of how Linux utilities that are used for networking purposes work. Finally, this project emphasized the importance of extensive testing and showed me how difficult it is to have an accurate yet easy to use testing tool.