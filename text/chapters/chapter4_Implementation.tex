\chapter{Implementation} \label{Implementation}

\section{Front End}

\subsection{The SCIONLab Website}

\begin{minipage}{0.45\textwidth}
In the configuration form for user-\acsp{AS} of \acs{SCIONLab}'s web interface, I added an additional text input field (as visible in figure \ref{User-AS Configuration Form}) that allows the user to add an upper limit for the bandwidth. If the user then clicks the \textbf{Create \acs{AS}} button, the input from that field is read into the data model. Updating the configuration of an existing \acs{AS} works similarly. The maximum allowed bandwidth limit is the default bandwidth, which is set by the \acs{SCIONLab} administrators.
\end{minipage} \hfill
\begin{minipage}{0.5\textwidth}
\begin{figure}[H]
	\includegraphics[width=\textwidth]{img/user_as_form.png}
	\centering
	\caption{User-AS Configuration Form}
	\label{User-AS Configuration Form}
\end{figure}
\end{minipage}

\newpage
\subsection{The Data Model}

The \acs{SCIONLab} server is based on the Django framework. Therefore \acs{SCIONLab}'s data model is based on Django's data model as shown in figure \ref{Django Model Dependency Diagram}. The bandwidth limit is stored as an attribute in the \textit{Link} class. As visible in figure \ref{Django Model Dependency Diagram}, the \textit{Host} class can access all of its interfaces and therefore also all of its links. The \textit{Host} class represents the \acs{VM}, container or physical machine hosting the \acs{SCION} services. Since the host can access all information we need to generate the \textit{link\_info.json} file, we hand a host object as an argument into the function \textit{generate\_link\_info(host)} of the file \textit{bandwidth\_config.py}. This function returns a dictionary which is used in the \textit{config\_tar.py} file to generate a \acs{JSON} file and add the \textit{link\_info.json} file to the tarball that is later deployed to the \acl{AP}.

\begin{figure}[H]
	\centering
	\includegraphics[width=0.85\textwidth]{img/Django_diagram.png}
	\caption{Django Model Dependency Diagram}
	\label{Django Model Dependency Diagram}
\end{figure}

\subsection{Attachment Point Configuration}

On the \aclp{AP} there will be a Python file called \textit{scionlab\_config.py}. This file is responsible for configuring the \acsp{AP} and to restart the \acs{SCION} services after the configurations changed. Therefore, whenever we restart the \acs{SCION} services, we also have to invoke the \textit{scionlab\_bw\_limiter} (with the \textit{-l} option) in order to modify the \acs{TC} configuration according to the current \textit{link\_info.json} file. It is important to note that \acs{TC} configurations get automatically reset after each reboot and therefore need to be re-enforced every time the \acs{AP} restarts.

\section{Back End}

The back end consists mainly of the Python files listed below. Additionally there are some \acs{JSON}-files that are used to either safe some settings or store some text that would take up too much space in the code and is therefore put into a separate file for the sake of readability. The Python files have the following functionalities:

\begin{enumerate}
\item[$\bullet$]\textit{scionlab\_bw\_limiter}:
\\
This file implements the entry point of the entire program. It parses command line arguments, let's you configure the default bandwidth and the path to the \textit{link\_info.json} file and invokes the \textit{bandwidth\_configurator.py} in order to enforce or reset bandwidth limitations or to show the current \acs{TC} configuration. It accepts the following options:
	\begin{enumerate}
	\item[\textit{-h}:] Show help text that explains how to use the program. This text is stored in the \textit{config\_files/help.json} file.
	\item[\textit{-l}:] Enforce the bandwidth limitations according to the \textit{link\_info.json} file.
	\item[\textit{-r}:] Reset any previously set \acs{TC} configurations.
	\item[\textit{-s}:] Show the current \acs{TC} configuration.
	\item[\textit{-b}:] Set or update the default bandwidth. This takes as an argument a positive integer, which represents the default bandwidth in kilo bits per second.
	\item[\textit{-p}:] Set or update the path to the \textit{link\_info.json} file. This option takes the path to that file as an argument.
	\end{enumerate}

\item[$\bullet$]\textit{code\_base/bandwidth\_configurator.py}:
\\
This file contains the implementation of the \textit{limit()}, \textit{reset()} and \textit{show()} functions. The \textit{limit()} function reads in the \textit{link\_info.json} file and creates link objects accordingly. Then it sets up the virtual interfaces, creates the \acs{TC} logic and invokes the \textit{make()} function from the \textit{tc\_logic.py} file. The \textit{reset()} function simply deletes the root \acs{QDISC} as well as the ingress \acs{QDISC} for each used interface. And finally the \textit{show()} function is responsible for printing out the current \acs{TC} configuration.

\item[$\bullet$]\textit{code\_base/cmd\_executor.py}:
\\
The command executor implements some static helper functions, that simplify running a command on the command line. One function silently runs a command using the subprocess \acs{API}, an other one runs a command and prints it to the console, one runs the command silently but returns the output it received by running the command and finally one function runs a command after printing it and returns the output.

\item[$\bullet$]\textit{code\_base/constants.py}:
\\
This file contains some static variables that are constant throughout the entire program. 

\item[$\bullet$]\textit{code\_base/interfaces.py}:
\\
This file is used to retrieve information about the interfaces configured on the host machine and bundle it together in interface objects.

\item[$\bullet$]\textit{code\_base/links.py}:
\\
Analogously to the \textit{interfaces.py} file, the \textit{links.py} file is used to bundle together information about links. This information is parsed form the \textit{link\_info.json} file. Furthermore, in this file, the virtual interfaces are set up and mapped to their corresponding physical counterpart.

\item[$\bullet$]\textit{code\_base/systeminfo.py}:
\\
The \textit{systeminfo.py} file is used to retrieve some information about the host like whether a file exists or which interface is used by default.

\item[$\bullet$]\textit{code\_base/tc\_command\_generator.py}:
\\
The \acs{TC} command generator is used to generate the \acs{TC} commands that are used in order to configure the system. The generation functions return these commands as strings. They are later executed using the command executor.

\item[$\bullet$]\textit{code\_base/tc\_logic.py}:
\\
This file is the centrepiece of the entire program. It defines the building blocks to build up the entire logic of our \acs{TC} hierarchy according to figure \ref{QDISC-Set-up}. The \ac{UML}-model in figure \ref{tc logic uml} visualizes the relevant building blocks and their attributes and functions. The most interesting function is probably the \textit{make()} function. It first uses the \acs{TC} command generator to get the command to turn it's properties into \acs{TC} configurations, executes the command using the command executor and then recursively calls \textit{make()} on the building blocks that are lower down in the hierarchy tree. Like this the entire \acs{TC} logic can be turned into \acs{TC} configuration on the host. 

\item[$\bullet$]\textit{code\_base/virtual\_interfaces\_manager.py}:
\\
The virtual interface manager is used to set up and delete the virtual interfaces and to map them to their physical counterparts.

\end{enumerate}

\begin{figure}[h]
	\centering
  	\includegraphics[width=\textwidth]{img/tc_logic_uml.png}
    \caption{UML-Model of \text{tc\_logic.py}}
    \label{tc logic uml}
\end{figure}

\newpage
\textit{ }
\newpage
\subsection{Configuration Example}

This example shows how the \textit{link\_info.json} file gets turned into \acs{TC} configuration using the \textit{scionlab\_bw\_limiter}. Listing \ref{link_info.json} shows the entries for two user-\acsp{AS}.  The first one is connected to the \acs{AP} via an \acs{IP}v6 connection, whereas the second user-\acs{AS} uses an \acs{IP}v4 connection. Both \acsp{AS} have a bandwidth limit of 500 Kbps and the default bandwidth is 1000 Kbps. Listing \ref{TC Configuration} shows how \acs{TC} gets configured accordingly. Note that the \textit{scionlab\_bw\_limiter} also needs to set up the virtual interfaces. But since this is not really dependant on the \textit{link\_info.json} file, I omitted this part for the sake of readability. 

\begin{lstlisting}[language=sh, caption = link\_info.json, captionpos=b, numbers=left, frame=single, breaklines=true, breakatwhitespace=true, showstringspaces=false, label=link_info.json]
{
    "1": {
        "AS_ID": 1,
        "IsUserAS": true,
        "Bandwidth": 500,
        "IP-Address": "fe80::20c:29ff:fe08:993c"
    },
    "2": {
        "AS_ID": 2,
        "IsUserAS": true,
        "Bandwidth": 500,
        "IP-Address": "192.168.17.129"
    }
}
\end{lstlisting}

\newpage

\begin{lstlisting}[language=sh, caption = TC Configuration, captionpos=b, numbers=left, frame=single, breaklines=true, breakatwhitespace=true, showstringspaces=false, label=TC Configuration, basicstyle=\normalsize]
# Create root QDISC on the physical interface
sudo tc qdisc add dev ens33 root handle 1: htb default 9999
# Create classes
sudo tc class add dev ens33 parent 1:0 classid 1:9999 htb rate 1000kbit ceil 1000kbit burst 5k prio 9999 mtu 1500
sudo tc class add dev ens33 parent 1:0 classid 1:1 htb rate 500kbit ceil 500kbit burst 5k prio 1 mtu 1500
sudo tc class add dev ens33 parent 1:0 classid 1:2 htb rate 500kbit ceil 500kbit burst 5k prio 2 mtu 1500
# Create classifier filters 
sudo tc filter add dev ens33 u32 match ip6 dst fe80::20c:29ff:fe08:993c/128 flowid 1:1
sudo tc filter add dev ens33 u32 match ip dst 192.168.17.129/32 flowid 1:2
# Create ingress QDISC on the physical interface
sudo tc qdisc add dev ens33 handle ffff: ingress
# Create redirect filters
sudo tc filter add dev ens33 parent ffff: protocol ip u32 match ip src 0.0.0.0/0 action mirred egress redirect dev ifb0
sudo tc filter add dev ens33 parent ffff: protocol ipv6 u32 match ip6 src ::0/0 action mirred egress redirect dev ifb0
# Create root QDISC on the virtual interface
sudo tc qdisc add dev ifb0 root handle 2: htb default 9999
# Create classes
sudo tc class add dev ifb0 parent 2:0 classid 2:9999 htb rate 1000kbit ceil 1000kbit burst 5k prio 9999 mtu 1500
sudo tc class add dev ifb0 parent 2:0 classid 2:1 htb rate 500kbit ceil 500kbit burst 5k prio 1 mtu 1500
sudo tc class add dev ifb0 parent 2:0 classid 2:2 htb rate 500kbit ceil 500kbit burst 5k prio 2 mtu 1500
# Create classifier filters
sudo tc filter add dev ifb0 u32 match ip6 src fe80::20c:29ff:fe08:993c/128 flowid 2:1
sudo tc filter add dev ifb0 u32 match ip src 192.168.17.129/32 flowid 2:2
\end{lstlisting}